\documentclass[10pt]{beamer}

\usetheme[%
    progressbar=frametitle,
    block=fill,
    numbering=fraction,
    footer=crumbs,
    sectionpage=numbered,
    subsectionpage=none,
    titleformattitle=smallcaps,
    titleformatsubtitle=smallcaps,
    %%% new options for UR2 theme:
    maincolor=red,
]{metropolis-ur2}

\usepackage{hyperref}
\hypersetup{
    colorlinks=true,
    linkcolor=.,
    filecolor=ur2Turquoise,
    urlcolor=ur2Bleu,
    pdftitle={M1 MAS - Python - Généralités},
    pdfpagemode=FullScreen,
}

% \usepackage{biblatex}
% \addbibresource{example.bib}

\usepackage{appendixnumberbeamer}
\usepackage{adjustbox}
\usepackage{booktabs}
\usepackage{fontawesome5}
\usepackage{graphicx}
\usepackage{array}
\usepackage{tabularx}
\usepackage{myminted}
% \usepackage[cache=false]{minted}

\title[M1 MAS -- Python -- Généralités]{CM 0 - Généralités}
\subtitle{Programmation Python -- Master 1 MAS}
\author{Romain Tavenard}
\date{2023}
\institute{%
\hypersetup{urlcolor=.}
\makebox[2.2ex][c]{\faEnvelope}\enspace\href{mailto:romain.tavenard@univ-rennes2.fr}{\texttt{romain.tavenard@univ-rennes2.fr}}\\%
% \makebox[2.2ex][c]{\faHome}\enspace\url{https://rtavenar.github.io/}%
}

\begin{document}

\maketitle

\begin{frame}{Format du cours}  
  \begin{itemize}
    \item 12 semaines
    \begin{itemize}
      \item 1h CM, 1h TD
    \end{itemize}
    \item Travail personnel à prévoir
    \begin{itemize}
      \item 2h par semaine en moyenne (donc + pour certain(e)s)
      \item Projets
      \item Préparation de cours + QCM
    \end{itemize}
    \item Évaluation
    \begin{itemize}
      \item 2 contrôles continus (1 papier, 1 sur machine)
      \item Projets
      \item Méthode de calcul de la note finale encore inconnue
    \end{itemize}
  \end{itemize}
\end{frame}

\begin{frame}{Contenu du cours}  
  \begin{itemize}
    \item 4 grandes parties
    \begin{enumerate}
      \item Bases du langage
      \item Structures de données (listes, dictionnaires, \emph{etc.})
      \item Accès aux données (fichiers, requêtes HTTP)
      \item Programmation Orientée Objet
    \end{enumerate}
  \end{itemize}
\end{frame}

\begin{frame}{Organisation (1/2)}  
  \begin{itemize}
    \item À chaque nouvelle partie
    \begin{itemize}
      \item Les CM + TD + Projet de la partie sont mis en ligne
      \item Pas de projet pour la partie 1
      \item Projet non noté (mais avec \emph{feedback}) pour la partie 2
    \end{itemize}
    \item Avant chaque CM
    \begin{itemize}
      \item Un ou plusieurs chapitres de cours à travailler (polycopié)
      \item Un QCM CURSUS à remplir
    \end{itemize}
    \item Après chaque TD
    \begin{itemize}
      \item Finir le TD par soi-même
    \end{itemize}
  \end{itemize}
\end{frame}

\begin{frame}{Organisation (2/2)}  
  \begin{itemize}
    \item En séance de CM
    \begin{itemize}
      \item Focus sur quelques points saillants du cours
      \item Venir avec ses questions sur le cours
      \item Venir avec ses questions sur le projet
    \end{itemize}
    \item En séance de TD, par ordre de priorité :
    \begin{enumerate}
      \item Le TD du jour
      \item Le(s) TD(s) suivant(s) de la partie en cours
      \item Le projet de la partie en cours
    \end{enumerate}
  \end{itemize}
\end{frame}

\begin{frame}{Dualité TD / Projet}  
  \begin{itemize}
    \item Pourquoi TD \alert{et} projet ?
    \begin{itemize}
      \item Hétérogénéité de niveau en Python (début d'année)
      \item Apprendre un langage de programmation implique de la répétition
      \item Python sera un (LE ?) langage central de vos études en Master
    \end{itemize}
    \item Principe de base :
    \begin{itemize}
        \item Je sais faire les TD sans problème $\Rightarrow$ je vise 10/20
        \item Je sais faire les TD et les projets sans problème $\Rightarrow$ je vise 20/20
    \end{itemize}  
  \end{itemize}
\end{frame}

% \begin{frame}{Table des Matières sur 2 colonnes}
% \twocol{\tableofcontents[sections={1-2}]}{\tableofcontents[sections={3-4}]}

% \end{frame}


% \section{Champ des possibles}

% \subsection{Usage}

% \begin{frame}[fragile]{Usage}
%   \begin{itemize}
%     \item Copiez le répertoire \texttt{sty/} dans votre projet.
  
%     \item Utilisez ce thème en incluant l'en-tête suivante dans votre document Beamer :\medskip
  
%     \begin{minted}{latex}
%     \usetheme[maincolor=red]{metropolis-ur2}
%     \end{minted}
    
%     \medskip
%     \item Plus précisément, ce document a été généré avec l'en-tête : \medskip
    
%     \begin{beamercodeblock}\vspace{-.6em}
%       \begin{minted}[linenos,fontsize=\footnotesize]{latex}
%         \usetheme[%
%             progressbar=frametitle, block=fill,
%             numbering=fraction, footer=crumbs,
%             sectionpage=numbered, subsectionpage=none,
%             titleformattitle=smallcaps, titleformatsubtitle=smallcaps,
%             %%% Options spécifiques au thème UR2:
%             maincolor=red]{metropolis-ur2}
%       \end{minted}
%     \end{beamercodeblock}
%     \beamercaptionblock{\textbf{Extrait de code :} Options de thème.}
%   \end{itemize}
% \end{frame}

% \subsection{Options du thème}

% {
% \metroset{maincolor=grey}
% \begin{frame}[fragile,stretch=3]{Options : Couleur}

%     \begin{itemize}
%     \item L'option \texttt{maincolor} permet de définir le jeu de couleurs à utiliser pour la diapositive en cours.
    
%     \item Les trois choix possibles sont \colorbox{ur2Rouge}{\textcolor{white}{\texttt{red}}} (par défaut), \colorbox{white}{\textcolor{black}{\texttt{white{\vphantom l}}}} ou \colorbox{ur2Gris}{\textcolor{white}{\texttt{grey{\vphantom l}}}}.

%     \item Ce diaporama utilise \texttt{maincolor=red} mais cette diapositive a été forcée à \texttt{maincolor=grey} via :\medskip

%           \mint{latex}|   \metroset{maincolor=grey}|
%     \medskip
%     (voir le code source de cet exemple pour plus d'infos)
%     \end{itemize}
% \end{frame}
% }

% \begin{frame}[stretch]{Options : Logos}
%   \begin{itemize}
%     \item L'option \texttt{titlelogo} définit le logo à insérer dans la diapositive de titre.  Les valeurs possibles sont :
%     \begin{itemize}
%         \item \texttt{none}: pas de logo
%         \item \texttt{urdeux}: le logo UR2 \alert{\bf (par défaut)}
%     \end{itemize}
%     \item L'option \texttt{headlogo} fait la même chose pour le logo de la ligne de titre des diapositives.
%   \end{itemize}
% \end{frame}

% \begin{frame}{Options : Divers}
%   En plus des options standard du thème \textsc{Metropolis}, ce thème utilise quelques options issues de \textsc{colorful-dream}:

%   \begin{itemize}
%     \item \texttt{footer=crumbs} indique la section / sous-section en cours dans le bas de page.
%     \item \texttt{sectionpage=numbered} génère une page de titre pour chaque section.
%     \item \texttt{subsectionpage=numbered} fait de même pour les \texttt{subsections} \textit{(pas activé dans cette présentation)}.
%   \end{itemize}
% \end{frame}

% \subsection{Layout}

% \begin{frame}[fragile,stretch=3.5]{Layout: Spacing}
% It's nice to have less text per slide!

% \begin{itemize}
%     \item Frames now have an additional \alert{\texttt{stretch}} key with a stretch factor as an optional value (defaults to~\texttt{2}).  It will \alert{increase spacing} between paragraphs and list items.

%     \item The idea is to make it easier to stretch slide contents, \emph{without} littering the code with \mintinline{latex}{\vspace{...}} commands everywhere.

%     \item This slide uses \texttt{stretch=3.5}.
% \end{itemize}

% \end{frame}

% \begin{frame}[fragile]{Layout: Two-column layout}
% To quickly get a two-column layout, you can use:\medskip
    
%     \mint{latex}|\twocol{First column here.}{Second column here.}|
%     \medskip
    
%     By default, this will make both columns equally wide, namely \mintinline{latex}{0.475\linewidth}.
    
%     \bigskip
%     \twocol[0.3]{An optional argument can be used to specify a different factor for the first column.  Here, I used \texttt{0.3}.}{The second column will automatically expand so that the two columns combined take up \mintinline{latex}{0.95\linewidth}. \textit{(This is less than~1 so that there is some padding between them.)}}
% \end{frame}


% \section{Colors}

% \begin{frame}{Colors}
%   This theme defines the following colors that you can use anywhere in your presentation. They are all based on the color values in JU's official Graphic Manual.
%   \bigskip
  
%   \centering
%   \begin{tabular}{ >{\raggedleft\arraybackslash} m{3cm} m{1cm}  >{\raggedleft\arraybackslash} m{3cm} m{1cm} }
%     \tt ur2Rouge & \colorbox{ur2Rouge}{\makebox(14,14){~}} &  \tt ur2Jaune & \colorbox{ur2Jaune}{\makebox(14,14){~}} \\
%     \tt ur2Gris & \colorbox{ur2Gris}{\makebox(14,14){~}} & \tt ur2Turquoise & \colorbox{ur2Turquoise}{\makebox(14,14){~}} \\
%     &&
%     \tt ur2Bleu & \colorbox{ur2Bleu}{\makebox(14,14){~}} \\
%     &&
%     \tt ur2GrisClair & \colorbox{ur2GrisClair}{\makebox(14,14){~}} \\
%   \end{tabular}
% \end{frame}

% \subsection{Usage}

% \begin{frame}[fragile]{Using Colors}

% Colors can be used with any commands, for example:
% \begin{center}
%     \begin{tabular}{ll}
%     \toprule
%         \mintinline{latex}{\textcolor{ur2Jaune}{Lorem ipsum}} & \textcolor{ur2Jaune}{Lorem ipsum} \\
%     \bottomrule
%     \end{tabular}
% \end{center}

% \medskip
% Some predefined commands that use these colors:

% \begin{center}
%     \begin{tabular}{ll}
%     \toprule
%         \mintinline{latex}{\alert{Lorem ipsum}} & \alert{Lorem ipsum} \\
%         \mintinline{latex}{\alertExample{Lorem ipsum}} & \alertExample{Lorem ipsum} \\
%         \mintinline{latex}{\highlight{Lorem ipsum}} & \highlight{Lorem ipsum} \\

%     \bottomrule
%     \end{tabular}
% \end{center}

% \medskip
% The colors are also used in different predefined \hyperlink{environments}{\textcolor{ur2Bleu}{environments}} and \hyperlink{blocks}{\textcolor{ur2Bleu}{blocks}}.

% \end{frame}

% \section{Environnements}

% \subsection{Énumérations}

% {\metroset{itemize=colored}
% \setbeamertemplate{enumerate items}[mycircle]
% \begin{frame}[fragile,stretch,label=environments]{Enumerate environments}
%     Combine \mintinline{latex}{\metroset{itemize=colored}} with
%     {\mintinline{latex}{\setbeamertemplate{enumerate items}[mycircle]}} to get:
% \medskip

%     \begin{enumerate}
%         \item One
%         \begin{enumerate}[a]
%             \item alpha
%             \begin{enumerate}[i]
%                 \item foo
%                 \item bar
%                 \item baz
%             \end{enumerate}
%             \item omega
%         \end{enumerate}
%         \item Two
%         \item Three
%     \end{enumerate}
% \end{frame}
% }

% \subsection{Quotations}

% \begin{frame}[fragile]{Quotations}

%     \begin{quote}%
%     Beautiful is better than ugly.
%     Explicit is better than implicit.
%     Simple is better than complex. [\ldots]
%     Readability counts.
%     \end{quote}
%     \attribution{Tim Peters, from \textsc{The Zen of Python}}

% \medskip
% The quotation above can be produced via:

%     {\footnotesize
%     \begin{minted}{latex}
%     \begin{quote}
%       Beautiful is better than ugly.
%       Explicit is better than implicit.
%       Simple is better than complex. [\ldots]
%       Readability counts.
%     \end{quote}
%     \attribution{Tim Peters, from \textsc{The Zen of Python}}
%     \end{minted}
%     }

% \end{frame}

% \begin{frame}[standout]
%    This is a standout slide.
%    \bigskip\bigskip
   
%    \normalsize
%    Standout slides are a Metropolis feature activated through the \texttt{[standout]} frame option.
%    \bigskip
   
%    In contrast to Metropolis defaults, here it will have the logotype headline, footer, and frame numbering.
% \end{frame}


% \subsection{Blocks}

% {\metroset{block=transparent}
% \begin{frame}[label=blocks]{Blocks}
%     These are beamer blocks with \texttt{block=transparent}.\medskip

%     \begin{block}{This is a regular \texttt{block}.}
%     Here is some content.
%     \end{block}

%     \begin{alertblock}{This is an \texttt{alertblock}.}
%     Here is some content.
%     \end{alertblock}

%     \begin{exampleblock}{This is an \texttt{exampleblock}.}
%     Here is some content.
%     \end{exampleblock}

%     \begin{warningblock}{This is a \texttt{warningblock}.}
%     Here is some content.
%     \end{warningblock}
% \end{frame}
% }

% {\metroset{block=fill}
% \begin{frame}{Blocks}
%     These are beamer blocks with \texttt{block=fill}.\medskip

%     \begin{block}{This is a regular \texttt{block}.}
%     Here is some content.
%     \end{block}

%     \begin{alertblock}{This is an \texttt{alertblock}.}
%     Here is some content.
%     \end{alertblock}

%     \begin{exampleblock}{This is an \texttt{exampleblock}.}
%     Here is some content.
%     \end{exampleblock}

%     \begin{warningblock}{This is a \texttt{warningblock}.}
%     Here is some content.
%     \end{warningblock}
% \end{frame}
% }

% \subsection{Code}
% \begin{frame}[fragile,stretch]{Code Blocks}
    
%     Since I frequently need to show code examples, I defined some styles and custom commands that can be included with:
    
%     \mint{latex}|\usepackage{myminted}|
    
%     This uses the \href{http://tug.ctan.org/macros/latex/contrib/minted/minted.pdf}{{\small\faExternalLink*}~\texttt{minted}} package to typeset code with automatic syntax highlighting.
%     \medskip

%     \begin{warningblock}{\faExclamationTriangle{}\enspace{}Important}
%     Every frame that contains code must have the \texttt{[fragile]} option set, or compilation will break with many cryptic errors!
%     \end{warningblock}
% \end{frame}

% \begin{frame}[fragile]{Code Blocks}
%     I also defined some commands to easily produce code blocks like the following; see the \LaTeX{} source for details:\medskip
    
%     \begin{beamercodeblock}\vspace{-.6em}
%       \begin{minted}[linenos,fontsize=\footnotesize]{java}
%       // Your First Program

%       class HelloWorld {
%         public static void main(String[] args) {
%           System.out.println("Hello, World!"); 
%         }
%       }
%       \end{minted}
%     \end{beamercodeblock}
%     \beamercaptionblock{\textbf{Listing 1:} A hello world program in Java.}

% \end{frame}


% \subsection{Bibliography}
% \begin{frame}[fragile,stretch]{Bibliography}
    
%     I use \texttt{biblatex} with some custom definitions, which I also bundled in their own package:
    
%     \mint{latex}|\usepackage{mybiblatex}|
    
%     Mainly, this will suppress output of URLs and DOIs, and instead turn the paper title into hyperlinks.  Here's an example:

%     \begin{itemize}
%         \item \fullcite{iki-aizawa-2020-language}
%     \end{itemize}
% \end{frame}

% \appendix
% \metroset{sectionpage=none}
% \section{Bibliography}

% \begin{frame}{Bibliography}

%     \nocite{*}
%     \printbibliography[heading=none]
    
%     This slide also demonstrates that appendices work \& play nicely with the other features!
% \end{frame}

% \begin{frame}[plain,standout,noframenumbering]
% \urLogoW[height=2.5cm]
% \end{frame}

\end{document}
