\documentclass[10pt]{beamer}

\usetheme[%
    progressbar=frametitle,
    block=fill,
    numbering=fraction,
    footer=crumbs,
    sectionpage=numbered,
    subsectionpage=none,
    titleformattitle=smallcaps,
    titleformatsubtitle=smallcaps,
    %%% new options for UR2 theme:
    maincolor=red,
]{metropolis-ur2}

\usepackage{hyperref}
\hypersetup{
    colorlinks=true,
    linkcolor=.,
    filecolor=ur2Turquoise,
    urlcolor=ur2Bleu,
    pdftitle={M1 MAS - Python - Structures de données},
    pdfpagemode=FullScreen,
}

% \usepackage{biblatex}
% \addbibresource{example.bib}

\usepackage{appendixnumberbeamer}
\usepackage{adjustbox}
\usepackage{booktabs}
\usepackage{fontawesome5}
\usepackage{graphicx}
\usepackage{array}
\usepackage{tabularx}
\usepackage{myminted}
% \usepackage[cache=false]{minted}

\title[M1 MAS -- Python -- numpy]{CM 5 - Les librairies numpy et scipy}
\subtitle{Programmation Python -- Master 1 MAS}
\author{Romain Tavenard}
\date{2025}
\institute{%
\hypersetup{urlcolor=.}
\makebox[2.2ex][c]{\faEnvelope}\enspace\href{mailto:romain.tavenard@univ-rennes2.fr}{\texttt{romain.tavenard@univ-rennes2.fr}}\\%
% \makebox[2.2ex][c]{\faHome}\enspace\url{https://rtavenar.github.io/}%
}

\begin{document}

\maketitle

\begin{frame}[fragile]{Présentation des librairies}  
  Calcul numérique en Python :
  \begin{itemize}
    \item \mintinline{python}|numpy| : calculs numériques de base (opérations arithmétiques, \dots)
    \item \mintinline{python}|scipy| : fonctions avancées (calculs de distance, Transformée de Fourier, résolution de systèmes d'équations, \emph{etc.})
  \end{itemize}

  \pause

  Ces opérations déjà implémentées sont \alert{beauuuuuuuuucoup} plus rapides que si vous les codiez vous-même. \\
  \alert{But du jeu :} se reposer sur les opérateurs existants tant que possible \\
  $\Rightarrow$ \alert{NE PAS} coder des boucles sur les indices de tableaux \mintinline{python}|numpy|

\end{frame}

\begin{frame}[fragile]{Le type numpy.ndarray}  
  Le type de base des objets manipulés est \mintinline{python}|numpy.ndarray| (tableau à $n$ dimensions) :
  \begin{itemize}
    \item 1 dimension : vecteur
    \item 2 dimensions : matrice
    \item 3 dimensions et plus : tenseur
  \end{itemize}

  \alert{Le terme dimension ici n'a pas le même sens qu'en maths !}

  \begin{beamercodeblock}
    \begin{minted}[fontsize=\footnotesize]{python}
      print(x.ndim)
    \end{minted}
  \end{beamercodeblock}

  Pour obtenir la taille d'un \mintinline{python}|numpy.ndarray| : 

  \begin{beamercodeblock}
    \begin{minted}[fontsize=\footnotesize]{python}
      print(x.shape)  # Affiche un tuple
    \end{minted}
  \end{beamercodeblock}
\end{frame}

\begin{frame}[fragile]{Exercice (1/2)}
  Soit la matrice suivante :

  \begin{beamercodeblock}
    \begin{minted}[fontsize=\footnotesize]{python}
      import numpy as np
      m = np.array([[1, 2, 3],
                    [0, 0, 0],
                    [9, 8, 7]])
      u = np.array([[6, 2, 3],
                    [0, 4, 1],
                    [0, 8, 7]])
    \end{minted}
  \end{beamercodeblock}

  \bigskip

  \begin{quote}
    Que vaut \mintinline{python}|m + 1| ? \\
    Que vaut \mintinline{python}|m + u| ? \\
    Que vaut \mintinline{python}|m * 1| ? \\
    Que vaut \mintinline{python}|m * u| ? \\
    Que vaut \mintinline{python}|m @ u| ?
  \end{quote}
\end{frame}

\begin{frame}[fragile]{Exercice (2/2)}
  Soit la matrice suivante :

  \begin{beamercodeblock}
    \begin{minted}[fontsize=\footnotesize]{python}
      import numpy as np
      m = np.array([[1, 2, 3],
                    [9, 8, 7]])
    \end{minted}
  \end{beamercodeblock}

  \bigskip

  \begin{quote}
    Que vaut \mintinline{python}|m.sum()| ? \\
    Que vaut \mintinline{python}|m.sum(axis=0)| ? \\
    Que vaut \mintinline{python}|m.sum(axis=1)| ?
  \end{quote}
\end{frame}

\begin{frame}[fragile]{Indiçage (1/3)}
  On peut utiliser les mêmes stratégies d'indiçage que pour les listes : 

  \begin{beamercodeblock}
    \begin{minted}[fontsize=\footnotesize]{python}
      import numpy as np
      m = np.array([[1, 2, 3],
                    [9, 8, 7]])

      print(m[0, :])     # Équivalent à print(m[0])
      print(m[:, 0])
      print(m[1:, :-1])
    \end{minted}
  \end{beamercodeblock}
\end{frame}

\begin{frame}[fragile]{Indiçage (2/3)}
  Petite spécificité \mintinline{python}|numpy| :

  \begin{beamercodeblock}
    \begin{minted}[fontsize=\footnotesize]{python}
      import numpy as np
      v = np.array([1, 2, 3])
      w = np.array([0, 10])

      print(v.shape)              # (3, )
      print(v[:, None].shape)     # (3, 1)
      print(v[None, :].shape)     # (1, 3)
    \end{minted}
  \end{beamercodeblock}
\end{frame}

\begin{frame}[fragile]{Indiçage (3/3)}
  Indiçage avec \mintinline{python}|None| : pour quoi faire ???

  \begin{beamercodeblock}
    \begin{minted}[fontsize=\footnotesize]{python}
      mat = np.zeros((3, 2))
      for i in range(len(v)):
        for j in range(len(w)):
          mat[i, j] = v[i] + w[j]
      
      # Remplacé par :
      mat = v[:, None] + w[None, :]
      #     ^ (3, 1)     ^ (1, 2)
    \end{minted}
  \end{beamercodeblock}

  $$
    \left(
      \begin{matrix}
        v_0 + w_0 & v_0 + w_1 \\
        v_1 + w_0 & v_1 + w_1 \\
        v_2 + w_0 & v_2 + w_1
      \end{matrix}
    \right)
  $$

  Cette astuce fonctionne pour n'importe quelle dimension.
\end{frame}

\begin{frame}[fragile]{Exercices}
  \begin{quote}
    \begin{enumerate}
      \item Si \mintinline{python}|m| est de \emph{shape} \mintinline{python}|(3, 2)| et que \mintinline{python}|n| est de \emph{shape} \mintinline{python}|(3, 4)|, 
    à quoi ressemblera \mintinline{python}|m[:, None, :] + n[:, :, None]| ?
    
      \item Écrivez une fonction qui prend en entrée deux vecteurs numpy et retourne la matrice des distances Euclidiennes paires à paires de leurs éléments :
    
    $$
    \left(
      \begin{matrix}
        d(v_0, w_0) & d(v_0, w_1) \\
        d(v_1, w_0) & d(v_1, w_1) \\
        d(v_2, w_0) & d(v_2, w_1)
      \end{matrix}
    \right)
  $$
    \end{enumerate}
  \end{quote}
\end{frame}

\end{document}
