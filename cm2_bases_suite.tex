\documentclass[10pt]{beamer}

\usetheme[%
    progressbar=frametitle,
    block=fill,
    numbering=fraction,
    footer=crumbs,
    sectionpage=numbered,
    subsectionpage=none,
    titleformattitle=smallcaps,
    titleformatsubtitle=smallcaps,
    %%% new options for UR2 theme:
    maincolor=red,
]{metropolis-ur2}

\usepackage{hyperref}
\hypersetup{
    colorlinks=true,
    linkcolor=.,
    filecolor=ur2Turquoise,
    urlcolor=ur2Bleu,
    pdftitle={M1 MAS - Python - Bases},
    pdfpagemode=FullScreen,
}

% \usepackage{biblatex}
% \addbibresource{example.bib}

\usepackage{appendixnumberbeamer}
\usepackage{adjustbox}
\usepackage{booktabs}
\usepackage{fontawesome5}
\usepackage{graphicx}
\usepackage{array}
\usepackage{tabularx}
\usepackage{myminted}
% \usepackage[cache=false]{minted}

\title[M1 MAS -- Python -- Bases (suite)]{CM 2 - Bases (suite)}
\subtitle{Programmation Python -- Master 1 MAS}
\author{Romain Tavenard}
\date{2025}
\institute{%
\hypersetup{urlcolor=.}
\makebox[2.2ex][c]{\faEnvelope}\enspace\href{mailto:romain.tavenard@univ-rennes2.fr}{\texttt{romain.tavenard@univ-rennes2.fr}}\\%
% \makebox[2.2ex][c]{\faHome}\enspace\url{https://rtavenar.github.io/}%
}

\begin{document}

\maketitle

\section{Compléments sur les boucles \& fonctions}

\begin{frame}[fragile]{Boucles}  
  Il est possible de ``perturber'' le comportement d'une boucle (\mintinline{python}|for| ou \mintinline{python}|while|) à l'aide d'une des instructions \mintinline{python}|break| ou \mintinline{python}|continue| :

  \begin{beamercodeblock}
    \begin{minted}[fontsize=\footnotesize]{python}
      for i in range(10):
        if i == 5:
          break
        print(i)
      
      for i in range(10):
        if i == 5:
          continue
        print(i)
    \end{minted}
  \end{beamercodeblock}
\end{frame}

\begin{frame}[fragile]{Fonctions}  
  Contexte :

  \begin{itemize}
    \item On doit coder deux fonctions
    \item L'une des deux est un cas particulier de l'autre
  \end{itemize}

  \begin{beamercodeblock}
    \begin{minted}[fontsize=\footnotesize]{python}
    def fonction_generale(truc, machin, chose):
      # Ici le corps de la fonction
      return quelque_chose
    
    def fonction_cas_particulier(truc)
      return fonction_generale(truc=truc, machin=32, chose="abc")
    \end{minted}
  \end{beamercodeblock}
\end{frame}

\begin{frame}{Exercice}
  \begin{quote}
    \begin{enumerate}
      \item Écrivez une fonction qui prend en entrée une chaîne de caractères \mintinline{python}|chaine| et un entier \mintinline{python}|n| et affiche les mots de la chaîne \mintinline{python}|chaine| les uns après les autres. \\ Attention : si un mot de longueur exactement égale à \mintinline{python}|n| est rencontré, il faudra afficher ce mot puis s'arrêter (et donc ne pas afficher les mots suivants).
      \item En utilisant la fonction précédemment codée, écrivez une fonction qui prend en entrée une chaîne de caractères \mintinline{python}|chaine| et affiche les mots de cette chaîne les uns après les autres. \\ Attention : si un mot de longueur exactement égale à \alert{5} est rencontré, il faudra afficher ce mot puis s'arrêter (et donc ne pas afficher les mots suivants).
    \end{enumerate}
  \end{quote}
\end{frame}

\section{Les modules en Python}

\begin{frame}[fragile]{Importer un module}  
  Variantes de syntaxe pour l'import :

  \begin{beamercodeblock}
    \begin{minted}[fontsize=\footnotesize]{python}
      import mon_module
      [...]
      print(mon_module.machin)

      import mon_module as truc  # Ici, on renomme
      [...]
      print(truc.machin)

      from mon_module import machin  # Ici, on n'importe que `machin`
      [...]
      print(machin)
    \end{minted}
  \end{beamercodeblock}
\end{frame}

\begin{frame}[fragile]{Un module, c'est quoi ?}
  
  \pause

  \begin{itemize}
    \item Un (ensemble de) fichier(s) Python
    \item Stocké où ?
    \pause
    \begin{itemize}
      \item Dans un dossier qui stocke tous les modules installés (\mintinline{python}|pip install ...|)
      \item Dans le dossier courant \\ (fichier \mintinline{python}|mon_module.py| $\Rightarrow$ module \mintinline{python}|mon_module|)
    \end{itemize}
    \item Que se passe-t-il lors de l'import ?
    \pause
    \begin{itemize}
      \item Le fichier correspondant au module est exécuté \\ (presque équivalent à \emph{``Run Python File''} dans VS Code)
    \end{itemize}
  \end{itemize}
\end{frame}

\begin{frame}[fragile]{Exercice}
  \begin{beamercodeblock}
    \begin{minted}[fontsize=\footnotesize, label=cm2.py]{python}
      import math

      print(math.pi)
    \end{minted}
  \end{beamercodeblock}
  
  \bigskip

  \begin{quote}
    Qu'est-ce qui pourrait selon vous expliquer que, en exécutant le fichier ``cm2.py'' ci-dessus, on obtienne l'erreur :
  
    \begin{beamercodeblock}
      \begin{minted}[fontsize=\footnotesize]{python}
        Traceback (most recent call last):
          File "cm2.py", line 3, in <module>
            print(math.pi)
        AttributeError: module 'math' has no attribute 'pi'
      \end{minted}
    \end{beamercodeblock}
  \end{quote}
\end{frame}


\end{document}
