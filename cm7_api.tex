\documentclass[10pt]{beamer}

\usetheme[%
    progressbar=frametitle,
    block=fill,
    numbering=fraction,
    footer=crumbs,
    sectionpage=numbered,
    subsectionpage=none,
    titleformattitle=smallcaps,
    titleformatsubtitle=smallcaps,
    %%% new options for UR2 theme:
    maincolor=red,
]{metropolis-ur2}

\usepackage{hyperref}
\hypersetup{
    colorlinks=true,
    linkcolor=.,
    filecolor=ur2Turquoise,
    urlcolor=ur2Bleu,
    pdftitle={M1 MAS - Python - Accès aux données},
    pdfpagemode=FullScreen,
}

% \usepackage{biblatex}
% \addbibresource{example.bib}

\usepackage{appendixnumberbeamer}
\usepackage{adjustbox}
\usepackage{booktabs}
\usepackage{fontawesome5}
\usepackage{graphicx}
\usepackage{array}
\usepackage{tabularx}
\usepackage{myminted}
% \usepackage[cache=false]{minted}

\title[M1 MAS -- Python -- Fichiers]{CM 7 - L'interaction avec des données en ligne}
\subtitle{Programmation Python -- Master 1 MAS}
\author{Romain Tavenard}
\date{2025}
\institute{%
\hypersetup{urlcolor=.}
\makebox[2.2ex][c]{\faEnvelope}\enspace\href{mailto:romain.tavenard@univ-rennes2.fr}{\texttt{romain.tavenard@univ-rennes2.fr}}\\%
% \makebox[2.2ex][c]{\faHome}\enspace\url{https://rtavenar.github.io/}%
}

\begin{document}

\maketitle

\section{Requêtes HTTP GET}

\begin{frame}[fragile]{Filtrage des résultats}


  \begin{beamercodeblock}
    \begin{minted}[fontsize=\footnotesize]{python}
      import requests

      url = # ...
      contenu = requests.get(url, params="age=20")
    \end{minted}
  \end{beamercodeblock}

  \begin{beamercodeblock}
    \begin{minted}[fontsize=\footnotesize]{python}
      import requests

      def filtre_20ans(donnees):
        return [d for d in donnees if d["age"] == 20]

      url = # ...
      contenu = requests.get(url)
      post_filtre = filtre_20ans(contenu)
    \end{minted}
  \end{beamercodeblock} 

  \begin{itemize}
    \item Quelle version préférer ? \pause \alert{La première}
  \end{itemize}

\end{frame}

\begin{frame}[fragile]{En-tête HTTP}
  Se faire passer pour un navigateur :
  \begin{beamercodeblock}
    \begin{minted}[fontsize=\footnotesize]{python}
      http_headers = {'User-Agent': "Mozilla/5.0"}
      contenu_brut = requests.get(url, headers=http_headers)
      contenu_json = contenu_brut.json()
    \end{minted}
  \end{beamercodeblock}

  Utile pour contourner certains filtres basiques des fournisseurs de données.
\end{frame}

\begin{frame}{Clés d'API}
  De nombreux fournisseurs de données demandent une clé d'API :
  \begin{itemize}
    \item Sorte d'identifiant-mot de passe
    \item Pour suivre les usages
    \item Pour limiter / facturer certains usages
  \end{itemize}

  $\Rightarrow$ Dans ce cas, créer la clé en amont puis la spécifier comme paramètre lors de chaque appel

  \pause

  \alert{Cette clé ne doit JAMAIS apparaître en clair dans votre code (lue dans un fichier externe)}

\end{frame}

% https://geoservices.ign.fr/documentation/services/api-et-services-ogc/calcul-altimetrique-rest

\begin{frame}{Exercice collectif}
  \begin{quote}
    Prendre connaissance de l'API IGN de calcul d'altitude et l'utiliser (dans le navigateur puis via Python) pour calculer les altitudes des points suivants :
    \begin{itemize}
      \item Samoëns : \mintinline{python}| {'lon': 6.73333, 'lat': 46.083328}|
      \item Rennes : \mintinline{python}| {'lon': -1.68333, 'lat': 48.083328}|
    \end{itemize}
  \end{quote}

  {\centering
    \url{https://geoservices.ign.fr/documentation/services/services-geoplateforme/altimetrie\#72673}
  }

\end{frame}

\end{document}
